%%%%%%%%%%%%%%%%%%%%%%%%%%%%%%%%%%%%%%%%%%%%%%%%%%%%%%%%%%%%%%%%%%%
%% 
%% Weiwu Zhu's resume
%%   - based off work by Michael DeCorte 
%%
%%%%%%%%%%%%%%%%%%%%%%%%%%%%%%%%%%%%%%%%%%%%%%%%%%%%%%%%%%%%%%%%%%%



%%
%% The following code sets up the document formatting
%%

%this assumes that res_yy.sty is in some path
\documentstyle[hyperref, margin, line]{res_yy}

\hypersetup{backref,pdfpagemode=Full,colorlinks=true,backref}

\addtolength{\oddsidemargin}{-0.45in}
\addtolength{\voffset}{-0.30in}
\addtolength{\textwidth}{1.00in} \addtolength{\textheight}{1.50in}

\renewcommand{\namefont}{\LARGE\emph}



%%
%% The following code defines some macros for terms which have raised font
%% (ie 4\fourth would result 4th with the 'th' raised (superscripted)
%%

\def\Cplusplus{{\rm C\raise.5ex\hbox{\small ++}}}
\def\CSharp{{\rm C\raise.5ex\hbox{\small \#}}}
% 'st' 'nd' 'rd' 'th' superscripts for numbers
\def\first{{\raise.5ex\hbox{\small st}}}
\def\second{{\raise.5ex\hbox{\small nd}}}
\def\third{{\raise.5ex\hbox{\small rd}}}
\def\fourth{{\raise.5ex\hbox{\small th}}}



%%
%% starting the actual document
%%

\begin{document}

%the name in big fonts at the top of resume
%this is left aligned
\name{Weiwu Zhu}

%this is right aligned
\address{
%No. 5 Dan Ling St. Haidian District, Beijing, 100080, PRC \ \ \ \ \ +86-158-5005-3751 (CN),  weiwuzhu@hotmail.com 
19039 84th PL NE, Bothell, WA, 98011 \ \ \ \ \ \ \ \ \ \  (206) 886-5178 \ \ \ \ \ \ \ \ \ \  weiwuzhug@gmail.com
}

\begin{resume}

%%\section{\textsc{Objective}}
%%Seek a position of Software Development Engineer

\vspace*{-20pt}
\section{\textsc{Education}}

\textbf{Peking University, Beijing, China} \hfill Sep. 2008 - Jun. 2011 \\
M.S. in Software Engineering\\
\newline
\textbf{Wuhan University, Wuhan, China} \hfill Sep. 2004 - Jun. 2008 \\
B.S. in Software Engineering\\

%\begin{formatb}
 % \employer{l}\dates{r}\\
%  \body\\
%\end{formatb}
%\vspace*{3pt}
%\section{\textsc{Awards and Honors}}
%\employer{\textbf{Annual Von Neumann Award}}
%\dates{ Jun. 2011}
%\begin{position}
%Excellent graduate student
%\end{position}
%\vskip -0.2 cm
%\employer{\textbf{Microsoft Research Asia}}
%\dates{ Nov. 2010}
%\begin{position}
%Certificate of Excellent Internship
%\end{position}
%\vskip -0.2 cm
%\employer{\textbf{Annual Euler Award}}
%\dates{ Jun. 2009}
%\begin{position}
%Excellent graduate student
%\end{position}
%\vskip -0.2 cm
%\employer{\textbf{Singapore-Beijing Modeling Program}}
%\dates{Jul. 2006}
%\begin{position}
%Meritorious Winner
%\end{position}
%\vskip -0.2 cm
%\employer{\textbf{China Mathematical Modeling Contest, Beijing}}
%\dates{Sep. 2005}
%\begin{position}
%Outstanding Winner
%\end{position}
%\vskip -0.2 cm
%\textbf{China SINOPEC Scholarship} \hfill 2004 - 2005
%\vskip -0.2 cm
%\employer{\textbf{21st College Physics Contest, Beijing}}
%\dates{Dec. 2004}
%\begin{position}
%3rd Place Award
%\end{position}
%\vskip -0.2 cm
%\employer{\textbf{China National Scholarship}}
%\dates{2003 - 2004}
%\begin{position}
%1st Place Award
%\end{position}

%%
%% the meat of the resume starts now
%%

\begin{formatb}
  \employer{l}\title{r}\\
  \location{l}\dates{r}\\
  \body\\
\end{formatb}

\vspace*{-8pt}
\section{\textsc{Employment}}

\employer{\textbf{Microsoft}}
\title{Senior Software Engineer}
\location{Bellevue, WA, United States}
\dates{Jan. 2015 - Present}
\begin{position}
\vspace*{-8pt}
\begin{itemize}
\item Working on Bing \textbf{Local Entity Search} as the \textbf{Tech Lead}, especially on entity understanding by leveraging \textbf{Deep Learning}, search relevance, and large scale data mining.
\end{itemize} 
\end{position}

\employer{\textbf{Microsoft China}}
\title{Software Engineer}
\location{Beijing, China}
\dates{Jul. 2011 - Dec. 2014}
\begin{position}
\vspace*{-8pt}
\begin{itemize}
\item Worked on \textbf{Knowledge Graph} cataloging billions of entities and their relationships, and drove Microsoft next generation \textbf{Answer Platform} based on the Knowledge Graph as the \textbf{Tech Lead}, and enabled inexperienced users to create and ship a new Answer experience on \textbf{Bing} in few hours efficiently.
\vspace*{-12pt}
\newline
\item Worked on the backend algorithm of \textbf{Speech Synthesis} (\textbf{TTS}) engine and \textbf{Natural Language Generation} (\textbf{NLG}) engine for Microsoft \textbf{Cortana}, \textbf{Xbox}, and Nokia Here turn-by-turn navigation.
\end{itemize} 
\end{position}

%%\employer{\textbf{Speech Group at Microsoft China}}
%%\title{Software Development Engineer}
%%\location{Beijing, China}
%%\dates{Jul. 2011 - Jul. 2013}
%%\begin{position}
%%\vspace*{-4pt}
%%\begin{itemize}
%%\item Responsible for the development and maintainance of the backend algorithm of speech synthesis (TTS) engine for Microsoft Conversational Understanding (CU) systems and all %%other products having speech output.
%%\item Projects: Xbox CU Netflix, Windows Phone 8, Nokia TBT Navigation, Windows 8.1
%%\item Technique: rich unit selection, non-uniform unit selection, statistical parameter synthesis, hidden markov model (HMM), decision tree, C++
%%\end{itemize} 
%%\end{position}

\begin{formatb}
  \employer{l}\title{r}\\
  \location{l}\dates{r}\\
  \body\\
\end{formatb}

\section{\textsc{Experience}}

\employer{\textbf{Speech Group at Microsoft Research Asia}}
\title{Intern Researcher}
\location{Beijing, China}
\dates{Jan. 2010 - Jun. 2011 }
\begin{position}
\vspace*{-8pt}
\begin{itemize}
\item Research project: Voice Activity Detection for speech recognition in noisy environments.
\vspace*{-2pt}
\item Research project: Speech Enhancement using microphone array for speech recognition.
\end{itemize} 
\end{position}

%%\employer{\textbf{Community Group at Sogou}}
%%\title{Intern Developer}
%%\location{Beijing, China}
%%\dates{Jul. 2009 - Sep. 2009}
%%\begin{position}
%%\vspace*{-4pt}
%%\begin{itemize}
%%\item Responsible for the development of Sogou Navigation website (http://123.sogou.com), which had more than 1M page view per day at that time.
%%\end{itemize}
%%\end{position}

%%\employer{\textbf{Lotus Division at IBM China}}
%%\title{Intern Developer}
%%\location{Beijing, China}
%%\dates{Jul. 2008 - Sep. 2008}
%%\begin{position}
%%\vspace*{-4pt}
%%\begin{itemize}
%%\item Responsible for development of Lotus Quicker 8.1 in IBM blue pathway internship program.
%%\end{itemize}
%%\end{position}
%%
%%\employer{\textbf{Peking University}}
%%\title{Teaching Assistant}
%%\location{Beijing, China}
%%\dates{Mar. 2009 - Jul. 2009}
%%\begin{position}
%%\vspace*{-4pt}
%%\begin{itemize}
%%\item Responsible for the design and development of music search engine by humming.
%%\end{itemize}
%%\end{position}


%%
%% We use the same formatting for projects as for work experience
%% Shown below is the formatting used previously
%%
%%  \begin{formatb}
%%    \employer{l}\title{r}\\
%%    \location{l}\dates{r}\\
%%    \body\\
%%  \end{formatb}
%%
%% 
%%  Note that \location is now being used for non-location information
%%


%\begin{formatb}
 % \employer{l}\dates{r}\\
  %\body\\
%\end{formatb}

\section{\textsc{Patent}}
\textbf{Weiwu Zhu}, jointly with Ming Sui \& Shengcai Peng, Language Modeling For Speech Recognition Leveraging Knowledge Graph, {\it WO2016196320}\\
\vspace*{-4pt}
\newline
\textbf{Weiwu Zhu}, jointly with Albert J. Thambiratnam \& Frank T. Seide, Leveraging speech recognition feedback for voice activity detection, {\it United States Patent No. 8,650,029}\\

\vspace*{-8pt}
\section{\textsc{Publication}}
\textbf{Weiwu Zhu}, Using Knowledge Graph and Search Query Click Logs in Statistical Language Model For Speech Recognition, {\it InterSpeech 2017}\\
\vspace*{-6pt}
\newline
Kit Thambiratnam, \textbf{Weiwu Zhu}, and Frank Seide, Voice Activity Detection Using Speech Recognizer Feedback, {\it InterSpeech 2012}\\


%%
%% This section could also use more formatting, but looks ok, as is
%%

%\section{\textsc{Qualifications}}

%\emph{Programming Languages}: \Cplusplus, \CSharp, Cg, HLSL, ARB assembly, SML, OCaML, PHP, MySQL, Java, Python, Perl, MIPS assembly

%\emph{Libraries and Tools}: Vim, STL, DirectX, OpenGL, \LaTeX, GIMP, Adobe Suite, Macromedia Suite, MatLab, Mathematica, Microsoft Visual Studio, GCC, GDB


%%
%% Note that we're redefining the formatting
%% We only have one row of information now, instead of two
%%

\vspace*{-12pt}
\section{\textsc{Award}}
\textbf{Hack Day in Microsoft Search Technology Center Asia, China} \hfill Apr. 2013 \\
Category Winner\\
\newline
\textbf{Microsoft Research Asia, China} \hfill Jul. 2011 \\
Certificate of Excellent Internship\\
\newline
\textbf{Peking University, China} \hfill 2008 - 2009 \\
School Scholarship Award\\
\newline
\textbf{Programming Contest in Wuhan University, China} \hfill 2007 - 2008 \\
3rd Place Award\\

\vspace*{-12pt}
\section{\textsc{Skill}}
\textbf{Programming}: Skilled in Python, and familiar with C++, C\#, Java, and SQL language; Familiar with Object Oriented Programming, Speech Technology, Knowledge Graph, Search Relevance, Design Patterns, Machine Learning, Operating System and Compiler;  \\
\vspace*{-6pt}
\newline
\textbf{Language}: Chinese (native), English (fluent)

%%
%% Nothing special here, just a normal table
%%

%\section{\textsc{Coursework}}
 % \begin{tabular}{lllll}
%Bayesian Networks   & \ \ & Machine Learning    & \ \ & Data Compression \\ 
%Computer Vision      & \ \ & Signal Processing      & \ \ & Programming Languages \\
%Algorithms        & \ \ & Data Structure    & \ \ & Artificial Intelligence\\
%Operation Research      & \ \ & Optimization Methods          & \ \ & Mathematical Statistics \\
%Numerical Analysis      & \ \ & Scientific Computing        & \ \ & Partial Differential Equations      \\
%Probability Theory     & \ \ & Stochastic Process       & \ \ & Time Series      \\
%\end{tabular}
\end{resume}


\end{document}

